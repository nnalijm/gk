\documentclass[a4paper, 12pt]{article}

\usepackage{graphicx}
\usepackage{xcolor}
\usepackage{listings}
\usepackage[T1]{fontenc}
\usepackage[utf8]{inputenc}
\usepackage[polish]{babel}


% image folder path
\graphicspath{{./pix/}}

% change tabel of contents label
\renewcommand\contentsname{Spis treści}

% listings configuration
\lstset{language=C++,
	basicstyle=\footnotesize,
	breakatwhitespace=false,
	breaklines=true,
	commentstyle=\color{green},
	frame=single,
	keepspaces=false,
	keywordstyle=\color{blue},
	numbers=left,
	numbersep=5pt,
	showspaces=false,
	showtabs=false,
	stringstyle=\color{red},
	tabsize=2,
}


% document's beginning
\begin{document}
\begin{titlepage}
\begin{center}
		\vspace*{1cm}
		\textbf{Grafika komputerowa\\Laboratorium\\}
		\vspace{2cm}
		\textbf{Stanislau Antanovich \& Mykola Sharonov}
		\vfill
		\vspace{0.8cm}
		\includegraphics[scale=0.7]{logo.png}
\end{center}
\end{titlepage}

\tableofcontents
\newpage

\section{Budowa obiektu sterowanego}
\subsection{Opis zadania}
Należy zbudować ``robot rolniczy (łazik)'' wykorzystując wyłącznie prymitywy bazujące na trójkącie. Obiekt ten będzie wykorzystywany na kolejnych zajęciach. W tworzonej grze komputerowej użytkownik będzie miał możliwość sterowania tym łazikiem.
\subsection{Wymagania}
Wymagania dotyczące budowy głósnego obiektu:
\begin{itemize}
\item Na ocenę 3:
Obiekt złożony z co najmniej 10 brył elementarnych (walec, prostopadłościan, itp.) zbudowanych przy użyciu prymitywów bazujących na trójkącie.
\item Na ocenę 4:
Obiekt złożony z co najmniej 20 brył elementarnych (walec, prostopadłościan, itp.) zbudowanych przy użyciu prymitywów bazujących na trójkącie.
\item Na ocenę 5:
Obiekt złożony z co najmniej 25 brył elementarnych (walec, prostopadłościan, itp.) zbudowanych przy użyciu prymitywów bazujących na trójkącie oraz projekt napisany obiektowo w C++.

Możliwość zaimportowania łazika z programu graficznego (np. Blender) o budowie odpowiadającej co najmniej 25 bryłom elementarnym.
\end{itemize}

\subsection{Realizaja zadania}

\section{Budowa otoczenia}
\subsection{Opis zadania}
Należy zbudować elementy otoczenia, w którym będzie poruszał się robot rolniczy wykorzystując wyłącznie prymitywy bazujące na trójkącie. Elementy te będą wykorzystywane na kolejnych zajęciach i będą powiązanie z fabułą gry.
\subsection{Wymagania}
Wymagania dotyczące budowy otoczenia:
\begin{itemize}
\item Na ocenę 3:
Przygotowanie otoczenia o podłożu płaskim oraz utworzenie dwóch obiektów dodatkowych (drzewo, bramka, budynek).
\item Na ocenę 4:
Przygotowanie otoczenia o podłożu nieregularnym (góra, stadion, wyboista ziemia) oraz utworzenie jednego obiektu dodatkowego.
\item Na ocenę 5:
Import otoczenia z programu graficznego (otoczenie o podłożu nieregularnym i minimum 1 obiekt dodatkowy).
\end{itemize}

\subsection{Realizaja zadania}

\section{Teksturowanie}
\subsection{Opis zadania}

Należy dokonać teksturowania według przedstawionych poniżej kryteriów.
\subsection{Wymagania}

Wymagania dotyczące dodania teksurowania.
\begin{itemize}
\item Na ocenę 3: Teksturowanie obiektów otoczenia oraz utworzenie autorskiego rozwiązania sterowaniem kamerą.
\item Na ocenę 4: Jak na ocenę 3 oraz teksturowanie powierzchni.
\item Na ocenę 5: Jak na ocenę 4 oraz teksturowanie obiektu, który będzie sterowany (minimum 3 bryły).
\end{itemize}
\subsection{Realizacja zadania}

\section{Sterowanie obiektem głównym}
\subsection{Opis zadania}
Należy dokonać sterowanie obiektem głównym.
\subsection{Wymagania}
Wymagania dotyczące sterowania obiektem głównym.
\begin{itemize}
\item Na ocenę 3: Realizacja prostego sterowanie przód-tył i obrót wokół własnej osi.
\item Na ocenę 4: Implementacja prostej fizyki sterowania (w przypadku łazika różnica prędkości na gąsienicach lub oś skrętna).
\item Na ocenę 5: Jak na ocenę 4 oraz implementacja podstawowych zagadnień fizycznych np. pęd ciała.
\end{itemize}
\subsection{Realizacja zadania}
\end{document}
